

%<!-- Titleseite -->
\thispagestyle{empty}

%%% redefine \maketitle
\renewcommand{\maketitle}{
	\begin{titlepage}
	
	\Huge
	\begin{center}
		Deliberation Across the World\\ 
		{\large
			A Cross-National Examination of the Link Between Deliberation and Regime Legitimacy
		}
		\hspace{5cm}
		
		% Author names and affiliations
		\Large
		Rosa Seitz$^1$, Fabio Votta$^2$ \\
		
		\hspace{8pt}
		
		\normalsize  
		$^1$) Graduate Student at the University of Stuttgart\\
		\texttt{rosa.marie.seitz@gmail.com}\\
		\textmd{+49 173 9900909}\\
		$^2$) Graduate Student at the University of Stuttgart\\
		\texttt{fabio.votta@gmail.com}\\
		\textmd{+49 174 2021316}\\
	\end{center}
	
	\hspace{5pt}
	
	\small
	
	A core assumption of deliberative theory is that deliberation helps generate legitimate decisions in the political sphere and beyond.  In this context, this paper seeks to investigate whether deliberation increases citizens perception of regime legitimacy, which is conceptualized and measured as regime support.
	
	To this end, deliberation and its relationship with regime support are examined across the world by combining multiple cross-national survey projects (World Values Survey, Asian Barometer, Afrobarometer, Latinobarómetro, AmericasBarometer and European Social Survey) covering 119 countries and over 316,938 individual respondents. As this paper is the first known to the authors that examines the effects of deliberation on regime support in a cross-country design, the used deliberation measurement, the Deliberative Component Index from the "Varieties of Democracy"-Project, is thoroughly examined and analyses are conducted for its components as well. Given that self-reported regime support is expected to be biased in countries where freedom of discussion is inhibited, a weight is applied to account for possible distortions in the survey data.
	
	
	The results of the multilevel regression analysis indicate that deliberation fulfills its legitimacy claim by increasing regime support first and foremost in democracies. The evidence for non-democracies and the complete sample is ambiguous and less robust although it points in the same direction. A possible limitation of the study is the high correlation between the used deliberation measurements and democracy indices which might indicate that the measurement does not accurately reflect the grade of deliberation in each country. Nevertheless, some interesting deviations in these correlations could be found between the subsamples, as well as in regard to the index and its components. In order to arrive at robust results, more sensible ways to measure deliberation on the country level seem necessary. It is further suggested that following studies and survey projects in the field should find methods to remedy possible bias in self-reported regime support, especially in countries with more authoritarian regimes.
	
	The findings of this paper can hopefully be used as a starting point for more detailed investigation of the relationship between deliberation and regime legitimacy in the future.
	
	\scriptsize
	\hspace{5pt}
		\begin{center}
			Note: Rosa Seitz and Fabio Votta are currently enrolled as graduate students of \textit{Empirical Political- and Social Research} at the University of Stuttgart. This paper is not published or accepted for publication at this time, although it will be presented at the 2018 ECPR General Conference in Hamburg (August 22-25).
			
			\hspace{3pt}
			
			Online Appendix can be accessed here: \url{https://favstats.github.io/delib_mod_database}
			
			In the interest of Open Science, the entire code that was used to generate the content of this paper can be found in the following GitHub Repository: \url{https://github.com/favstats/paper_delib}
		\end{center}
	

	\end{titlepage}
}

%%% automated table of contents
\newcommand{\contents}{
\newpage
\thispagestyle{empty}
\vspace{20mm}
\tableofcontents
}



%%% Title page
\maketitle
\newpage
%\contents
%\clearpage
%\listoffigures
%\clearpage
%\listoftables
%\clearpage	